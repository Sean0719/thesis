\chapter{Preliminaries and Background}
This section provides a broad overview of foundational ideas
and background material relevant to this thesis.
In most chapters of this thesis, we include a deeper
discussion of the related literature relevant to
that material.

\section{Preliminaries}
The content in this thesis builds on the following topics.
We assume preliminary knowledge of these topics and
give a limited set of key references here.
The reader should have an understanding of statistical
and machine learning modeling paradigms as described in
\citet{wasserman2013all,bishop2007pattern,friedman2001elements}.
Our contributions mostly focus on end-to-end modeling with
deep architectures as described in
\citet{schmidhuber2015deep,goodfellow2016deep} with
applications in computer vision as described in
\citet{forsyth2003modern,bishop2007pattern,szeliski2010computer}.
Our contributions also involve optimization theory and
applications as described in
\citet{bertsekas1999nonlinear,boyd2004convex,bonnans2013perturbation,griewank2008evaluating,nocedal2006sequential,sra2012optimization,wright1997primal}.
One application area of this thesis work focuses on control
and reinforcement learning.
Control is one kind of optimization-based modeling and
is further described in
\citet{bertsekas2005dynamic,sastry2011adaptive,levine2017optimal}.
Reinforcement learning methods are summarized in
\citet{sutton1998reinforcement,levine2017introduction}.

\section{Energy-based Learning}
Energy-based learning is a machine learning method
typically used in supervised settings that explicitly
adds relationships and dependencies to the model's
output space.
This is in contrast to purely feed-forward models that
typically cannot explicitly capture dependencies
in the output space.
At the core of energy-based learning methods is
a scalar-valued \emph{energy} function
$E_\theta(x,y): \mathcal{X}\times\mathcal{Y}\rightarrow \mathbb{R}$
parameterized by $\theta$ that measures the fit
between some input $x$ and output $y$.
Inference in energy-based models is done by
solving the optimization problem
\begin{equation}
  \label{eq:bg:inf}
  \hat y = \argmin_{y} E_\theta(x, y).
\end{equation}
We note that this is a powerful formulation for
modeling and learning and subsumes the representational
capacity of standard deep feedforward models,
which we show how to do in \cref{sec:bg:ff-energy}.
The energy function can also be interpreted
from a probabilistic lens as the negated unnormalized
joint distribution over the input and output spaces.

Energy-based methods have been in use for over a decade
and the tutorial \citet{lecun2006tutorial} overviews
many of the foundational methods and challenges
in energy-based learning.
The two main challenges for energy-based learning are
1) learning the parameters $\theta$ of the energy
function $E_\theta$ and 2) efficiently solving
the inference procedure in \cref{eq:bg:inf}.
These challenges have historically been tamed by
using simpler energy functions consisting of
hand-engineered feature extractors for the inputs
$x$ and linear functions of $y$.
This captures models such as Markov random fields
\citep{li1994markov}
and conditional random fields
\citep{lafferty2001conditional,sutton2012introduction}.
Standard gradient-based methods are difficult to use
for parameter learning because $\hat y$ depends on $\theta$
through the $\argmin$ operator, which is not always differentiable.
Historically, a common approach to doing parameter learning
in energy-based models has been to directly shape the
energy function with a max-margin approach
\citet{taskar2004max,taskar2005learning}.

More recently, there has been a strong push to further incorporate
structured prediction methods like conditional random fields as the
``last layer'' of a deep network architecture
\citep{peng2009conditional,zheng2015conditional,chen2015learning}
as well as in deeper energy-based architectures
\citep{belanger2016structured,belanger2017end,belanger2017deep},
which we further discuss in \cref{sec:bg:spens}.

An ongoing discussion in the community argues whether
adding the dependencies explicitly in an energy-based is useful or not.
Feedforward models have a remarkable representational
capacity that can implicitly learn the dependencies and
relationships from data without needing to impose
additional structure or modeling assumptions and
without making the model more computationally
expensive with an optimization-based inference procedure.
One argument against this viewpoint that supports
energy-based modeling is that explicitly including
modeling information improves the data efficiency
and requires less samples to learn because some
structure and knowledge is already present in the model
and does not have to be learned from scratch.

\subsection{Energy-based Models Subsume Feedforward Models}
\label{sec:bg:ff-energy}
We highlight the power of energy-based modeling for supervised
learning by noting how they subsume deep feedforward models.
Let $\hat y = f_\theta(x)$ be a deep feedfeedforward model.
The energy-based representation of this model is
$E(x,y) = ||y-f_\theta(x)||_2^2$
and inference becomes the convex optimization problem
$\hat y = \argmin_y E(x,y)$, which
has the exact solution $\hat y = f_\theta(x)$.
An energy function that has more structure over the output
space adds representational capacity that a feedforward
model wouldn't be able to capture explicitly.

\subsection{Structured Prediction Energy Networks}
\label{sec:bg:spens}
Structured Prediction Energy Networks (SPENs)
\citep{belanger2016structured,belanger2017end,belanger2017deep}
are a way of bridging the gap between modern deep
learning methods and classical energy-based learning
methods.
SPENs provide a deep structure over input and output spaces
by representing the energy function $E_\theta(x,y)$ with
a standard feed-forward neural network.
This expressive formulation comes at the cost of making
the inference procedure in \cref{eq:bg:inf} difficult
and non-convex.
SPENs typically use an approximate inference procedure
by taking a fixed-number of gradient descent steps
for inference.
For learning, SPENs typically replace the inference with
an \emph{unrolled gradient-based optimizer}
that starts with some prediction $y_0$ and
takes a fixed number of gradient steps to minimize
the energy function
$$y_{i+1} = y_i - \alpha \nabla_y E_\theta(x,y_i).$$
The final iterate as then taken as the prediction
$\hat y \triangleq y_N$.
Gradient-based parameter learning can be done by
differentiating the prediction $\hat y$ with respect
to $\theta$ by unrolling the inference procedure.
Unrolling the inference procedure
can be done in most autodiff frameworks such as
PyTorch \citep{paszke2017automatic}
or TensorFlow \citep{abadi2016tensorflow}.
activation functions with smooth first derivatives
such as the sigmoid or softplus \citep{glorot2011deep}
should be used to avoid discontinuities because
unrolling the inference procedure involves computing
$\nabla_\theta \nabla_y E_\theta(x,y)$.

\section{Modeling with Domain-Specific Knowledge}
\label{sec:bg:dsn}
The role of domain-specific knowledge in the machine learning
and computer vision fields has been an active discussion topic
over the past decade and beyond.
Historically, domain knowledge such as fixed hand-crafted
feature and edge detectors were rigidly part of the
computer vision pipeline and have been overtaken by
learnable convolutional models \citet{lecun1998mnist,krizhevsky2012imagenet}.
To highlight the power of convolutional architectures, they
provide a reasonable prior for vision tasks even without
learning \citep{ulyanov2018deep}.
Machine learning models extend far beyond the reach
of vision tasks and the community has a growing interest
on domain-specific priors rather than just using
fully-connected architectures.
These priors ideally can be integrated as end-to-end
learnable modules into a larger system that are
learned as a whole with gradient-based information.
In contrast to pure fully-connected architectures,
specialized submodules ideally improve the data
efficiency of the model, add interpretability,
and enable grey-box verification.

Recent work has gone far beyond the classic examples of
adding modeling priors by using convolutional or
sequential models.
A full discussion of all of the
recent improvements is beyond the scope of this thesis,
and here we highlight a few key recent
developments.
% \todon{Re-phrase some of these}
\begin{itemize}
\item Differentiable beam search \citep{goyal2018continuous}
  and differentiable dynamic programming \citep{mensch2018differentiable}
\item Differentiable protein simulator \citep{ingraham2018learning}
\item Differentiable particle filters \citep{jonschkowski2018differentiable}
\item Neural ordinary differential equations \citep{chen2018neural}
  and applications to reversible generative models \citep{grathwohl2018ffjord}
\item Relational reasoning on sets, graphs, and trees
  \citep{battaglia2018relational,zaheer2017deep,
    kipf2016semi,gilmer2017neural,santoro2017simple,
    hamilton2017inductive,battaglia2016interaction,
    xu2018powerful,farquhar2017treeqn,shen2018ordered}
\item Geometry-based priors
  \citep{bronstein2017geometric,gulcehre2018hyperbolic,
     monti2017geometric,tang2018ba,li2018smoothing}
\item Memory \citep{sukhbaatar2015end,graves2014neural,graves2016hybrid,xiong2016dynamic,hill2015goldilocks,parisotto2017neural}
\item Attention \citep{bahdanau2014neural,vaswani2017attention,wang2018non}
\item Capsule networks \citep{sabour2017dynamic,hinton2018matrix,xinyi2018capsule}
\item Program synthesis
\citep{reed2015neural,neelakantan2015neural,balog2016deepcoder,devlin2017robustfill,parisotto2016neuro}
\end{itemize}

\section{Optimization-based Modeling}
\label{sec:bg:opt}
Optimization can be used for modeling in machine learning.
Among many other applications, these architectures are well-studied for
generic classification and structured prediction tasks
\citep{goodfellow2013multi,stoyanov2011empirical,brakel2013training,lecun2006tutorial,belanger2016structured,belanger2017end};
in vision for tasks such as denoising
\citep{tappen2007learning,schmidt2014shrinkage}
or edge-aware smoothing \citep{barron2016fast}.
\citet{metz2016unrolled} uses unrolled optimization within a network to
stabilize the convergence of generative adversarial networks
\citep{goodfellow2014generative}.
Indeed, the general idea of solving restricted classes of
optimization problem using neural networks goes back
many decades \citep{kennedy1988neural, lillo1993solving},
but has seen a number of advances in recent years.
These models are often trained by one of
the following four methods.

\subsection{Explicit Differentiation}
If an analytic solution to the argmin can be found,
such as in an unconstrained quadratic minimization,
the gradients can often also be computed analytically.
This is done in \citet{tappen2007learning,schmidt2014shrinkage}.
We cannot use these methods for
the constrained optimization problems
we consider in this thesis because
there are no known analytic solutions.

\subsection{Unrolled Differentiation}
\label{sec:bg:unroll}
The argmin operation over an unconstrained objective can be approximated
by a first-order gradient-based method and unrolled.
These architectures typically introduce an optimization
procedure such as gradient descent into the inference procedure.
This is done in
\citet{domke2012generic,belanger2017end,metz2016unrolled,goodfellow2013multi,stoyanov2011empirical,brakel2013training,finn2017model}.
The optimization procedure is unrolled automatically or manually
\citep{domke2012generic} to obtain derivatives during training that incorporate
the effects of these in-the-loop optimization procedures.

Given an unconstrained optimization problem with a
parameterized objective
$$\argmin_x f_\theta(x),$$
gradient descent starts at an initial value $x_0$ and
takes steps $$x_{i+1} = x_i - \alpha \nabla_x f_\theta(x).$$
For learning, the final iterate of this procedure $x_N$ can
be taken as the output and
$\partial x_N/\partial \theta$ can
be computed with automatic differentiation.

In all of these cases, the optimization problem is unconstrained
and unrolling gradient descent is often easy to do.
When constraints are added to the optimization problem, iterative
algorithms often use a projection operator that may be difficult
to unroll through and storing all of the intermediate iterates
may become infeasible.

\subsection{Implicit argmin differentiation}
\label{sec:bg:argmin-diff}
Most closely related to this thesis work, there have been
several applications of the implicit function theorem
to differentiating through constrained convex argmin
operations.
These methods typically parameterize an optimization problem's
objective or constraints and then applies the
\emph{implicit function theorem} (\cref{theorem:implicit})
to optimality conditions
of the optimization problem that implicitly define
the solution, such as the \emph{KKT conditions}
\citep[Section~5.5.3]{boyd2004convex}.
We will first review the implicit function theorem
and KKT conditions and then discuss related work
in this space.

\emph{Implicit function} analysis
\citep{dontchev2009implicit}
typically focuses on solving an equation $f(p,x)=0$ for $x$ as
a function $s$ of $p$, \ie $x=s(p)$.
\emph{Implicit differentiation} considers how to
differentiate the solution mapping with respect
to the parameters, \ie $\nabla_p s(p)$.
The \emph{implicit function theorem} used in standard
calculus textbooks can be traced back to the
lecture notes from 1877-1878 of \citet{dini1877analisi}
and is presented in
\citet[Theorem~1.B.1]{dontchev2009implicit} as follows.

\begin{theorem}[Implicit function theorem]
  \label{theorem:implicit}
  Let $f: \RR^d \times \RR^n \rightarrow \RR^n$ be continuously
  differentiable in a neighborhood of
  $(\bar p, \bar x)$ and such that $f(\bar p, \bar x)=0$,
  and let the partial Jacobian of $f$ with respect to
  $x$ at $(\bar p, \bar x)$, namely $\nabla_x f(\bar p, \bar x)$,
  be nonsingular. Then the solution mapping
  $S(p) = \menge{x\in \RR^n}{f(p,x)=0}$ has a single-valued
  localization $s$ around $\bar p$ for $\bar x$ which
  is continuously differentiable in a neighborhood $Q$
  of $\bar p$ with Jacobian satisfying
  $\nabla s(p) = -\nabla_x f(p, s(p))^{-1} \nabla_p f(p, s(p))$
  for every $p\in Q$.
\end{theorem}

In addition to the content in this thesis, several other papers
apply the implicit function theorem to differentiate through
the argmin operators.
This approach frequently comes up in bilevel optimization
\citep{gould2016differentiating,kunisch2013bilevel}
and sensitivity analysis
\citep{bertsekas1999nonlinear,fiacco1990sensitivity,bell2008algorithmic,bonnans2013perturbation}.
\citep{barratt2018differentiability} is a note on applying the
implicit function theorem to the KKT conditions of convex
optimization problems and highlights assumptions behind the
derivative being well-defined.
\citet{gould2016differentiating} describes general techniques for
differentiation through optimization problems,
but only describe the case of exact equality constraints rather than
both equality and inequality constraints
(they add inequality constraints via a barrier function).
\citet{johnson2016composing} performs implicit differentiation on
(multi-)convex objectives with coordinate subspace constraints.
The older work of \citet{mairal2012task} considers argmin
differentiation for a LASSO problem, derives specific rules for this case, and
presents an efficient algorithm based upon our ability to solve
the LASSO problem efficiently.
\citet{jordan2015convex} studies convex optimization over probability
measures and implicit differentiation in this context.
\citet{bell2008algorithmic} adapts automatic differentiation
to obtain derivatives of implicitly defined functions.

\subsection{An optimization view of the ReLU, sigmoid, and softmax}
\label{sec:bg:existing}
In this section we note how the commonly used ReLU, sigmoid, and softmax
functions can be interpreted as explicit closed-form solutions
to constrained convex optimization (argmin) problems.
We use these as examples to further highlight the power of
optimization-based inference, not to provide a new analysis
of these layers.
In this thesis, we propose new optimization-based inference
layers that do \emph{not} have explicit closed-form solutions
like these examples and show that they can still be efficiently
turned into differentiable building blocks for end-to-end architectures.

\begin{theorem}
  The ReLU, defined by $f(x) = \max\{0, x\}$,
  can be interpreted as projecting a point $x\in\RR^n$ onto
  the non-negative orthant as
  \begin{equation}
    f(x) = \argmin_y \;\; \frac{1}{2}||x-y||_2^2 \;\; \st \;\; y\geq 0.
    \label{eq:relu-proj}
  \end{equation}
\end{theorem}

\begin{proof}
  The usual solution can be obtained by looking at
  the KKT conditions of \cref{eq:relu-proj}.
  Introducing a dual variable $\lambda\geq 0$ for the inequality
  constraint, the Lagrangian of \cref{eq:relu-proj} is
  \begin{equation}
    L(y, \lambda)=\frac{1}{2}||x-y||_2^2 - \lambda^\top y.
  \end{equation}
  The stationarity condition
  $\nabla_y L(y^\star, \lambda^\star) = 0$
  gives a way of expressing the primal optimal
  variable $y^\star$ in terms of the dual optimal
  variable $\lambda^\star$ as $y^\star=x+\lambda^\star$.
  Complementary slackness $\lambda^\star_i(x_i+\lambda^\star_i)=0$
  shows that $\lambda^\star_i\in\{0, -x_i\}$.
  Consider two cases:
  \begin{itemize}
  \item \textbf{Case 1:} $x_i\geq 0$.
    Then $\lambda^\star_i$ must be 0
    since we require $\lambda^\star\geq 0$.
    Thus $y^\star_i=x_i+\lambda^\star_i=x_i$.
  \item \textbf{Case 2:} $x_i< 0$.
    Then $\lambda^\star_i$ must be $-x_i$
    since we require $y\geq 0$.
    Thus $y^\star_i=x_i+\lambda^\star_i=0$.
  \end{itemize}
  Combining these cases gives the usual solution of
  $y^\star=\max\{0, x\}$.
\end{proof}

\begin{theorem}
  The sigmoid or logistic function, defined by $f(x) = (1+e^{-x})^{-1}$,
  can be interpreted as projecting a point $x\in\RR^n$ onto
  the interior of the unit hypercube as
  \begin{equation}
    f(x) = \argmin_{0<y<1} \;\; -x^\top y -H_b(y),
    \label{eq:sigmoid-proj}
  \end{equation}
  where $H_b(y) = - \left(\sum_i y_i\log y_i + (1-y_i)\log (1-y_i)\right)$ is the
  binary entropy function.
\end{theorem}

\begin{proof}
  The usual solution can be obtained by looking at
  the first-order optimality condition of
  \cref{eq:sigmoid-proj}.
  The domain of the binary entropy function $H_b$ restricts
  us to $0<y<1$ without needing to explicitly represent this
  as a constraint in the optimization problem.
  Let $g(y; x) = -x^\top y -H_b(y)$ be the objective.
  The first-order optimality condition $\nabla_y g(y^\star; x) = 0$
  gives us $-x_i + \log y_i^\star - \log (1-y_i^\star) = 0$
  and thus $y^\star = (1+e^{-x})^{-1}$.
\end{proof}

\begin{theorem}
  The softmax, defined by $f(x)_j = e^{x_j} / \sum_i e^{x_i}$,
  can be interpreted as projecting a point $x\in\RR^n$ onto
  the interior of the $(n-1)$-simplex
  $$\Delta_{n-1}=\{p\in\RR^n\; \vert\; 1^\top p = 1 \; \; {\rm and} \;\; p \geq 0 \}$$
  as
  \begin{equation}
    f(x) = \argmin_{0<y<1} \;\; -x^\top y - H(y) \;\; \st\;\; 1^\top y = 1
    \label{eq:simplex-proj}
  \end{equation}
  where $H(y) = -\sum_i y_i \log y_i$ is the entropy function.
\end{theorem}

\begin{proof}
  The usual solution can be obtained by looking at
  the KKT conditions of \cref{eq:simplex-proj}.
  Introducing a scalar-valued dual variable $\nu$ for the
  equality constraint, the Lagrangian is
  \begin{equation}
    L(y, \nu) = -x^\top y - H(y) + nu(1^\top y - 1)
  \end{equation}
  The stationarity condition
  $\nabla_y L(y^\star, \lambda^\star) = 0$
  gives a way of expressing the primal optimal
  variable $y^\star$ in terms of the dual optimal
  variable $\nu^\star$ as
  \begin{equation}
    \label{eq:simplex-proj-pd}
    y^\star_j=\exp\{x_j-1-\nu^\star\}.
  \end{equation}
  Putting this back into the equality constraint
  $1^\top y^\star = 1$ gives us
  $\sum_i \exp\{x_i-1-\nu^\star\} = 1$ and thus
  $\nu^\star = \log\sum_i\exp\{x_i-1\}$.
  Substituting this back into \cref{eq:simplex-proj-pd}
  gives us the usual definition of
  $y_j = e^{x_j} / \sum_i e^{x_i}$.
\end{proof}

\begin{corollary}
A temperature-scaled softmax scales the entropy term in the objective
and the sparsemax \citep{martins2016softmax} replaces the
objective's entropy penalty with a ridge section.
\end{corollary}

\section{Reinforcement Learning and Control}
\label{sec:bg:rl}

The fields of reinforcement learning (RL) and optimal control
typically involve creating agents that act optimally
in an environment.
These environments can typically be represented
as a Markov decision process (MDP) with a continuous
or discrete state space and a continuous or discrete action space.
The environment often has some oracle-given
reward associated with each state and the goal of RL
and control is to find a policy that maximizes the
cumulative reward achieved.

Using the notation from \citep{levine2017introduction},
\emph{policy search} methods learn a policy $\pi_\theta(u_t|x_t)$
parameterized by $\theta$ that predicts a distribution over next
action to take given the current state $x_t$.
The goal of policy search is to find a policy that maximizes the
expected return
\begin{equation}
  \argmax_\theta\; \E_{\tau\sim p_\theta(\tau)} \left[ \sum_t \gamma^t r(x_t, u_t) \right],
\end{equation}
where $p_\theta(\tau)=p(x_1)\prod\pi_\theta(u_t|x_t)p(x_{t+1}|x_t,u_t)$
is the distribution over trajectories,
$\gamma\in(0,1]$ is a discount factor,
$r(x_t, u_t)$ is the state-action reward at time $t$,
and $p(x_{t+1}|x_t,u_t)$ is the state-transition probability.
In many scenarios, the reward $r$ is assumed to be
a black-box function that derivative information cannot
be obtained from.
\emph{Model-free} techniques for policy search typically do
not attempt to model the state-transition probability
while \emph{model-based} and \emph{control} approaches do.

\emph{Control approaches} typically provide a policy
by planning based on known state transitions.
For example, in continuous state-action spaces with deterministic
state transitions, the finite-horizon model predictive control problem is
\begin{equation}
    \argmin_{x_{1:T} \in \mathcal{X},u_{1:T}\in \mathcal{U}} \;\; \sum_{t=1}^T  C_t(x_t, u_t)
    \;\; \subjectto \;\; x_{t+1} = f(x_t, u_t), \;\; x_1 = \xinit,
\end{equation}
where $\xinit$ is the current system state,
the cost $C_t$ is typically hand-engineered and differentiable,
and $x_{t+1}=f(x_t, u_t)$ is the deterministic
next-state transition, \ie~the point-mass given by $p(x_{t+1}|x_t,u_t)$.
While this thesis focuses on the continuous and deterministic setting,
control approaches can also be applied in discrete
and stochastic settings.

\textbf{Pure model-free techniques for policy search} have
demonstrated promising results in many domains by learning
\emph{reactive polices} which directly map observations to actions
\citep{mnih2013playing,oh2016minecraft,gu2016continuous,lillicrap2015continuous,
schulman2015trust,schulman2016trpogae,gu2017qprop}.
Despite their success, model-free methods have many drawbacks and limitations,
including a lack of interpretability, poor generalization, and a
high sample complexity.
\textbf{Model-based methods} are known to be more sample-efficient
than their model-free
counterparts.
These methods generally rely on learning a dynamics model directly from
interactions with the real system and then integrate the learned model into the
control policy
\citep{schneider1997exploiting,abbeel2006using,deisenroth2011pilco,heess2015learning,
boedecker2014sparsegps}.
More recent approaches use a deep network to learn low-dimensional latent state
representations and associated dynamics models in this learned representation.
They then apply standard trajectory optimization methods
on these learned embeddings
\citep{lenz2015deepmpc, watter2015embed, levine2016end}.
However, these methods still require a manually specified and hand-tuned
cost function, which can become even more difficult in a latent representation.
Moreover, there is no guarantee that the learned dynamics model
can accurately capture portions of the state space relevant for the task at hand.

To leverage the benefits of both approaches, there has been significant
interest in \textbf{combining the model-based and model-free paradigms.}
In particular, much attention has been dedicated to utilizing
model-based priors to accelerate the model-free learning process.
For instance, synthetic training data can be generated by model-based control
algorithms to guide the policy search or prime a
model-free policy
\citep{sutton1990integrated,theodorou2010generalized,levine2014learning,
gu2016continuous,venkatraman2016improved,levine2016end,chebotar2017combining,
nagabandi2017mbmf, liting2017driving}.
\citep{bansal2017mbmf} learns a controller and then distills it to
a neural network policy which is then fine-tuned with model-free
policy learning.
However, this line of work usually keeps the model separate from the
learned policy.

Alternatively, the policy can include an \textbf{explicit planning module}
which \emph{leverages learned models} of the system or environment,
both of which are learned through model-free techniques.
For example, the classic Dyna-Q algorithm
\citep{sutton1990integrated} simultaneously learns a model of
the environment and uses it to plan.
More recent work has explored incorporating such structure into deep
networks and learning the policies in an end-to-end fashion.
\citet{tamar2016value} uses a recurrent network to predict the value function by
approximating the value iteration algorithm with convolutional layers.
\citet{karkus2017qmdp} connects a dynamics model to a planning
algorithm and formulates the policy as a structured recurrent network.
\citet{silver2016predictron} and \citet{oh2017value} perform multiple rollouts
using an abstract dynamics model to predict the value function.
A similar approach is taken by \citet{weber2017imagination} but directly
predicts the next action and reward from rollouts of an explicit environment model.
\citet{farquhar2017treeqn} extends model-free approaches, such as
DQN \citep{mnih2015human} and A3C \citep{mnih2016asynchronous}, by planning
with a tree-structured neural network to predict the cost-to-go.
While these approaches have demonstrated impressive results in
discrete state and action spaces, they are not applicable to
continuous control problems.

To tackle continuous state and action spaces, \citet{pascanu2017learning}
propose a neural architecture which uses an abstract environmental
model to plan and is trained directly from an external task loss.
\citet{pong2018temporal} learn goal-conditioned value functions and use them
to plan single or multiple steps of actions in an MPC fashion.
Similarly, \citet{pathak2018zero} train a goal-conditioned policy to perform
rollouts in an abstract feature space but ground the policy with a loss term
which corresponds to true dynamics data.
The aforementioned approaches can be interpreted as a distilled optimal controller
which does not separate components for the cost and dynamics.
Taking this analogy further, another strategy is to differentiate through an
optimal control algorithm itself.
\citet{okada2017path} and \citet{pereira2018pinets} present a way
to differentiate through path integral optimal control
\citep{williams2016aggressive,williams2017model}
and learn a planning policy end-to-end.
\citet{srinivas2018universal} shows how to embed
differentiable planning (unrolled gradient descent over actions) within
a goal-directed policy.
In a similar vein, \citet{tamar2017learning} differentiates
through an iterative LQR (iLQR) solver
\citep{li2004ilqr,xie2017ddp,tassa2014control}
to learn a cost-shaping term offline.
This shaping term enables a shorter horizon controller to approximate the
behavior of a solver with a longer horizon to save computation during runtime.

%%% Local Variables:
%%% coding: utf-8
%%% mode: latex
%%% TeX-engine: xetex
%%% TeX-master: "../thesis"
%%% End:
